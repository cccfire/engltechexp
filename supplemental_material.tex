% \documentclass[sigconf]{acmart}

% \usepackage{balance}
% \usepackage{subcaption}
% \usepackage{listings}
% \usepackage{cleveref}
% \usepackage{todonotes}

% % For custom enumerate and itemize labels:
% \usepackage{enumitem}

% % begin shadowimage

\usepackage{tikz}
\usetikzlibrary{shadows,calc}

% code adapted from https://tex.stackexchange.com/a/11483/3954

% some parameters for customization
\def\shadowshift{2pt,-2pt}
\def\shadowradius{6pt}

\colorlet{innercolor}{gray!60}
\colorlet{outercolor}{gray!05}

% this draws a shadow under a rectangle node
\newcommand\drawshadow[1]{
    \begin{pgfonlayer}{shadow}
        \shade[outercolor,inner color=innercolor,outer color=outercolor] ($(#1.south west)+(\shadowshift)+(\shadowradius/2,\shadowradius/2)$) circle (\shadowradius);
        \shade[outercolor,inner color=innercolor,outer color=outercolor] ($(#1.north west)+(\shadowshift)+(\shadowradius/2,-\shadowradius/2)$) circle (\shadowradius);
        \shade[outercolor,inner color=innercolor,outer color=outercolor] ($(#1.south east)+(\shadowshift)+(-\shadowradius/2,\shadowradius/2)$) circle (\shadowradius);
        \shade[outercolor,inner color=innercolor,outer color=outercolor] ($(#1.north east)+(\shadowshift)+(-\shadowradius/2,-\shadowradius/2)$) circle (\shadowradius);
        \shade[top color=innercolor,bottom color=outercolor] ($(#1.south west)+(\shadowshift)+(\shadowradius/2,-\shadowradius/2)$) rectangle ($(#1.south east)+(\shadowshift)+(-\shadowradius/2,\shadowradius/2)$);
        \shade[left color=innercolor,right color=outercolor] ($(#1.south east)+(\shadowshift)+(-\shadowradius/2,\shadowradius/2)$) rectangle ($(#1.north east)+(\shadowshift)+(\shadowradius/2,-\shadowradius/2)$);
        \shade[bottom color=innercolor,top color=outercolor] ($(#1.north west)+(\shadowshift)+(\shadowradius/2,-\shadowradius/2)$) rectangle ($(#1.north east)+(\shadowshift)+(-\shadowradius/2,\shadowradius/2)$);
        \shade[outercolor,right color=innercolor,left color=outercolor] ($(#1.south west)+(\shadowshift)+(-\shadowradius/2,\shadowradius/2)$) rectangle ($(#1.north west)+(\shadowshift)+(\shadowradius/2,-\shadowradius/2)$);
        \filldraw ($(#1.south west)+(\shadowshift)+(\shadowradius/2,\shadowradius/2)$) rectangle ($(#1.north east)+(\shadowshift)-(\shadowradius/2,\shadowradius/2)$);
    \end{pgfonlayer}
}

% create a shadow layer, so that we don't need to worry about overdrawing other things
\pgfdeclarelayer{shadow} 
\pgfsetlayers{shadow,main}


\newcommand\shadowimage[2][]{%
\begin{tikzpicture}
\node[anchor=south west,inner sep=0] (image) at (0,0) {\includegraphics[#1]{#2}};
\drawshadow{image}
\end{tikzpicture}}


% end shadowimage

% % set font for listings to courier
% \lstset{basicstyle=\ttfamily,breaklines=true}

% \begin{document}

\section{Supplemental Material}

\subsection{Expert Study}

Below, we describe the procedure for our expert study of Odyssey.

\subsubsection{Introduction and Background (8 min)}
Our process began with an introduction and background session. 
This phase involved introductions 
  followed by a set of background questions 
  aimed at understanding the participant's experience 
  and usage habits around numerical analysis tools. 
We asked about the participant's years of experience, 
  when they last analysed 
  the error of a floating-point expression, 
  their typical workflow 
  for analysing high floating-point error expressions, 
  and their familiarity with the Herbie tool. 
If the participant was not a user of Herbie, 
  we sought to understand their reasons for not using it and 
  asked if there were ways they imagined Herbie
  fitting into their workflow.

\subsubsection{Tutorial (12 min)}
Following the introductory phase, 
  we gave the participant access to the Odyssey interface
  via Zoom and conducted a twelve-minute tutorial
  to familiarize them 
  with the Herbie interface. 
The tutorial demonstrated several features 
  using the expression $\sqrt{x + 1} - \sqrt{x}$ 
  for positive $x$. 
The features covered included 
  the specification of the expression being rewritten and ranges over which it must be accurate, 
  reading the error plots, local error identification, 
  selecting expressions from the rewriting table,
  expression editing, opening a new expression in a different tab, 
  and resampling on a different range. 
Throughout this tutorial, 
  we encouraged participants 
  to think out loud and provide feedback, 
  emphasizing our interest 
  in continuous interface improvement.

\subsubsection{Tasks (30-55 min)}
The next phase of our process was a task-oriented session 
  whose length depended on participant skill and availability. 
The tasks were designed to exercise different 
  parts of the interface and to reveal insights 
  about the participants' understanding 
  and ability to apply Odyssey for expression rewriting.
The tasks covered 
  identifying sources of error 
  in specific mathematical expressions, 
  using the Odyssey system 
  to find and recommend improvements, 
  evaluating the effectiveness of proposed solutions, 
  and identifying problems in automated solutions. 
For each task, 
  specific goals were set ahead of time in terms of interface usage 
  and problem-solving approach so we could decide 
  if Odyssey was able to meet the participant's need 
  and whether their usage represented a novel approach.
Here is the full list of tasks and usage goals:
\begin{enumerate}
    \item \textbf{Identify relevant sources of error in the Rust $asinh$ implementation.}
        \begin{itemize}
        \item The participant should be able to determine the cause of the error by clicking on two different points to see at least two local error graphs.
        \end{itemize}
        
    \item \textbf{Use Odyssey to find and recommend $hypot$.}
        \begin{itemize}
        \item The participant should submit $sqrt(x * x + 1)$ in a new tab, ask Herbie for rewritings, and obtain $hypot(1, x)$ or another solution.
        \end{itemize}
        
    \item \textbf{Determine whether the solution to task 2 is good enough.}
        \begin{itemize}
        \item The participant should be able to integrate the result from task 2 into the original expression and refer to the error plot to justify their answer.
        \end{itemize}
        
    \item \textbf{Identify problems with branches (regimes) in automated solutions.}
        \begin{itemize}
        \item The participant should be able to highlight areas of concern by clicking on points around 1, where higher error is shown.
        \end{itemize}
        
    \item \textbf{Use Odyssey to find and recommend a way of solving small $x$ with $log1p$.}
        \begin{itemize}
        \item The participant should be able to find a good solution for the entire range of positive $x$ values that doesn’t include branches using Herbie's suggestions.
        \end{itemize}
        
    \item \textbf{Determine trust in the expression.}
        \begin{itemize}
        \item The participant should be able to verify the expression's equivalence to the original by checking the expression derivation.
        \end{itemize}
        
    \item \textbf{Use Odyssey to create a branched solution.}
        \begin{itemize}
        \item The participant should be able to create a branched expression that outperforms Herbie’s solution.
        \end{itemize}
    \end{enumerate}

\subsubsection{Survey and discussion (10-15 min)}
In the final phase, 
  we conducted a Google Forms survey that lasted between 10 to 15 minutes.
  The survey questions and results can be found in Table \ref{fig:survey-results}.

% \begin{itemize}
%   \item Introduction + background (8 min)
%   \begin{itemize}
%   \item Introductions
%   \item Background questions:
%   \begin{enumerate}
%   \item How many years of numerical analysis experience would you say you have?
%   \item When was the last time you analyzed the error of a floating-point expression (for a class or for your work)?
%   \item Sometimes short scalar floating-point expressions have high floating point error. What is your normal workflow for analyzing a mathematical expression you believe to have high floating-point error?
%   \item Have you used the Herbie floating-point error analysis tool (command line or web interface)?
%   \item (If they don’t use Herbie) Is there a reason you don’t use Herbie in this workflow? Any reflections on how Herbie could better fit your workflow?
%   \item Are there other tools you’ve worked with for solving numerical analysis problems?
%   \end{enumerate}
%   \end{itemize}
  
%   \item Tutorial (12 min)
%   \begin{itemize}
%   \item Get user set up with access to the interface via Zoom
%   \item Expression: $sqrt(x + 1) - sqrt(x)$ for positive $x$
%   \item Show features:
%   \begin{itemize}
%   \item Spec
%   \item Error plot – clickable
%   \item Local error – colored to indicate ``-'' is the problem here
%   \item Rewritings table
%   \item Editing field for entering new expressions
%   \item Ask Herbie for rewritings
%   \item Copying expressions into the edit box
%   \item Derivations
%   \item Zooming and resampling
%   \item Edit in new tab
%   \end{itemize}
%   \item ``While working on the tasks, please think out loud so we can understand what you are thinking about and trying to do. Please feel free to comment on the design at any time. There will also be time at the end of the interview for further discussion.''
%   \item ``We want to continue iterating on the design, so don't hold back any criticisms!''
%   \end{itemize}
  
%   \item Tasks (fit in as many as possible to exercise different parts of the interface) (30-55 min)
%   \begin{enumerate}
%   \item Identify relevant sources of error in Rust $asinh$
%   \begin{itemize}
%   \item Prompt
%   \begin{itemize}
%   \item $log(x + sqrt(x * x + 1) )$ for positive $x$ only is an expression for computing the inverse hyperbolic sine.
%   \item What is causing the error?
%   \item Follow-up: I am a grad student using this expression for a project. I only care about the behavior for relatively large values of $x$. Explain what I need to focus on fixing.
%   \end{itemize}
%   \item Usage goal
%   \begin{itemize}
%   \item Participant clicks on 2 different points to see at least 2 local error graphs, showing different problems with $sqrt$ and $log$
%   \item Participant clicks on a large value of $x$ to show that $sqrt$ has problems in the local error graph
%   \end{itemize}
%   \end{itemize}
%   \item Use Odyssey to find and recommend hypot *
% \begin{itemize}
%     \item Prompt
%     \begin{itemize}
%         \item ``I’m interested in a good way of rewriting the square root subexpression based on what you’ve said. Can you look at that subexpression by itself and try to get something better for that?''
%       \end{itemize}
%       \item Usage goal
%       \begin{itemize}
%       \item Participant submits $sqrt(x * x + 1)$ in a new tab
%       \item Participant asks Herbie for rewritings or uses own ideas to solve
%       \item Participant receives $hypot(1, x)$ or other solution
%       \end{itemize}
%       \end{itemize}

%       \item Is the solution to task 2 good enough?
% \begin{itemize}
%     \item Prompt
%     \begin{itemize}
%         \item ``Can you put that result back into the original expression?''
%         \item ``Assuming again that I only care about large values of $x$, is this good enough?''
%     \end{itemize}
%     \item Usage goal
%     \begin{itemize}
%         \item Participant copies the result into the original expression
%         \item Participant refers to the error plot to justify their answer (yes).
%     \end{itemize}
% \end{itemize}

% \item Identify problems with branches (regimes) in automated solutions
% \begin{itemize}
%     \item Prompt
%     \begin{itemize}
%         \item ``Try getting Herbie’s results for the current expression $log(x + hypot(1.0, x) )$. Note these results have branches.''
%         \item ``How would you feel about using these results in a major math library? Can you use the interface to show me an area of concern?''
%     \end{itemize}
%     \item Usage goal
%     \begin{itemize}
%         \item Participant clicks on points around 1 (the branch point, where higher error is shown)
%     \end{itemize}
% \end{itemize}

% \item Use Odyssey to find and recommend a way of solving small $x$ with $log1p$
% \begin{itemize}
%     \item Prompt
%     \begin{itemize}
%         \item ``You remember that a good way of handling the log issue is to use log1p, which is accurate for $x$ so small that $1 + x == 1$ in floating-point accuracy.''
%         \item ``Try opening a new tab with this expression: $log1p(x + hypot(x, 1) - 1 )$''
%         \item ``Where is the error now?''
%         \item ``Can you find a good solution for the entire range of positive $x$ values that doesn’t include branches?''
%     \end{itemize}
%     \item Usage goal
%     \begin{itemize}
%         \item Participant asks Herbie for suggestions
%     \end{itemize}
% \end{itemize}

% \item Do you trust the expression?
% \begin{itemize}
%     \item Prompt
%     \begin{itemize}
%         \item Do you trust that this expression is equivalent to the original expression?
%     \end{itemize}
%     \item Usage goal
%     \begin{itemize}
%         \item Participant checks the expression derivation.
%     \end{itemize}
% \end{itemize}

% \item Use Odyssey to create a branched solution
% \begin{itemize}
%     \item Prompt
%     \begin{itemize}
%         \item Start with $(exp(x) - 2) + exp(-x)$ for positive $x$
%         \item ``Can you create a solution that is better than anything Herbie can give you?''
%     \end{itemize}
%     \item Usage goal
%     \begin{itemize}
%         \item Participant creates a branched expression that beats Herbie’s solution
%     \end{itemize}
% \end{itemize}

% \end{enumerate}

% \item Survey (10-15 min)
% \begin{itemize}
% \item See Table~\ref{fig:survey-results} for survey questions and results.  %2
% \end{itemize}
% \end{itemize}

% \end{document}