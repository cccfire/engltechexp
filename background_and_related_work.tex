\section{Background and Related Work}
Odyssey draws on techniques from the developer tool literature on program visualization and program history
  to addresses key challenges developers face
  in the domain of floating-point arithmetic.

\subsection{Program Visualization for Debugging}

Floating-point error analysis and repair
  involves a mix of debugging and performance optimization work.
Odyssey is thus inspired by work aimed at
  program visualization for debugging.
Systems such as Whyline~\cite{ko2004designing}, Timelapse~\cite{burg2013interactive}, and FireCrystal~\cite{oney2009firecrystal},
  which connect code with runtime behavior by visualizing execution traces,
  inspire several of Odyssey's interactions,
  including the interactive ``local error'' heatmap 
  visualizing per-operation floating-point error for a particular input.
Moreover,
  a series of papers on integrating visualizations with code,
  such as Theseus~\cite{lieber2014theseus}, 
  which provides always-on visualizations of runtime state;
  Projection Boxes~\cite{lerner2020projectionBoxes}, 
  which gives programmers more control
  over which runtime values are visualized;
  and Hoffswell et al.~\cite{hoffswell2018augmentingCodeWithInSituVis}, 
  which provides recommendations
  for embedding visualizations in code,
  are reflected in our design of Odyssey's error graph,
  which allows programmers to visualize floating-point error
  and control which input values and rewritings
  are visualized.
Odyssey sees similar benefits from these designs as prior work:
  opening up space for programmer exploration and observation,
  and thereby giving programmers a fuller understanding of the problem space
  and a richer set of interactions for comparison and repair.

That said, floating-point rounding error
  is a continuous, numeric quality of a program,
  and the ``tuning'' stage of numerical work therefore has
  a lot of analogs to performance optimization.
Beck et al.~\cite{beck2013perfBottlenecksVis} and PerformanceHat~\cite{cito2018performanceHat}, for example,
  visualize the proportion of runtime
  spent at each each line of code in the program. 
  These approaches inspire our ``heatmap'' design for local error information,
  coloring each floating-point operation in the program
  based on the amount of floating-point rounding error
  it contributes to the result.
The Roly-poly~\cite{ikarashi2021guidedOptimization} project is also quite similar to Odyssey,
  aiding developers in exploring and selecting performance optimizations
  for image processing code.
Odyssey explores a similar system-aided optimization workflow,
  but for accuracy instead of performance optimization.


\subsection{Maintaining and Reviewing Code Versions}

To understand, experiment with, and collaborate on code,
  developers author and compare
  multiple program alternatives and histories~\cite{codoban2015softwareHistory}.
Tools such as Azurite~\cite{yoon2015azurite}, Verdant~\cite{kery2018verdant}, and Variolite~\cite{kery2017variolite}
  provide explicit support for multiple program versions.
For example, Verdant helps data scientists
  compare, replay, and simplify histories
  for code in computational notebooks~\cite{kery2018verdant}. 
  Also, Head et al.~\cite{head2019managingMesses} introduce ``code gathering'' techniques
  that find the minimal code slices in a program
  that produce a selected set of results. 
Comparing and combining multiple alternative rewritings
  is a also key part of floating-point error repair.

Odyssey maintains a history of rewritings
  both to provide a history of how a rewriting was developed
  and also allow developers to visualize, compare, and combine
  multiple alternatives, providing explicit internal support
  to what would otherwise be internal mental operations,
  thereby reducing cognitive load and allowing developers
  to focus on the higher-level problem-solving aspects.

\subsection{Floating-Point Arithmetic and Numerical Analysis}

Floating-point arithmetic,
  defined by the IEEE~754 standard~\cite{ieee08-standard},
  and variations of this standard
  form the standard number representation
  in most programming languages~\cite{toplas08-pitfalls-verifying}.
However, floating-point arithmetic is subject to rounding error,
  and even elementary computations often permit significant
  error~\cite{acm91-every-scientist}.
Numerical analysis provides a set of mathematical tools
  to analyze, bound, and reduce this error~\cite{book87-nmse}.
However, many programmers are unfamiliar with
  numerical analysis techniques, and even fewer have
  a thorough understanding of how to apply these tools.

Researchers have thus developed a vast menagerie of tools
  automating specific numerical analysis techniques,
  including Rosa~\cite{popl14-rosa} for affine arithmetic,
  FPTaylor~\cite{fm15-fptaylor} for error Taylor series, and
  Ariadne~\cite{popl13-ariadne} for root finding.
Other tools repurpose static analysis techniques
  to find floating-point rounding errors;
  such tools include Fluctuat~\cite{vmcai11-fluctuat},
  which uses abstract interpretation;
  FPDebug~\cite{pldi12-fpdebug},
  which uses a dynamic execution with shadow variables;
  and CGRS~\cite{ppopp14-cgrs}, which uses evolutionary search. 
These tools can find inputs with high rounding error
  or, in some cases, certify the absence of such errors.
Programmers can then use the error found
  to attempt to understand the source of the rounding error,
  and ultimately fix it.
One popular tool combining these steps is Herbie~\cite{herbie}.
Herbie uses sampling techniques to identify floating-point error;
  constructs candidate rewrites using algebraic and analytic identities,
  and tests those rewrites against higher-precision executions
  to identify the rewrite with the lowest floating-point error.
In recent releases, Herbie can output multiple suggestions
  with different performance and accuracy characteristics~\cite{pherbie}.

Unfortunately, all of these tools, Herbie included,
  are difficult for developers to use and integrate into their workflows.
Users are typically expected to
  identify the expression and inputs of interest up front;
  compare them to other sources or the user's own ideas;
  and make trade-offs between accuracy and other goals
  (e.g., maintainability),
  all without tool support.
Users are often recommended
  to switch between their code editor, version control system,
  a mathematical visualization tool, and multiple Herbie instances
  in order to solve a single problem~\cite{kneusel-numbers}.
VSCode-PRECiSA \cite{vscode-precisa}, 
  a VSCode interface for the PRECiSA 
  command-line tool \cite{precisa} designed to support the process 
  of analyzing a single program in several ways,
  is somewhat of an exception; 
  however, it does not address the problem of tool
  interoperation.
We developed Odyssey to address these limitations
  by providing a single integrated workbench
  for the full floating-point rounding error workflow.
To lower the barriers to adoption,
  Odyssey uses Herbie, a widely used and open source
  tool~\cite{herbieGithub,kneusel-numbers},
  under the hood. 


\subsection{Expert tools for design space search}
Odyssey is an expert tool for numerical analysts to re-write floating point
expressions. Unlike tools for end-users, expert tools are designed for users
with extensive design and implementation experience. Experts have honed
specialized workflows, leverage insights to improve upon automated or
semi-automated approaches, and are comfortable wading into low-level details.
For example, expert developers optimize the performance of
applications~\cite{cito2018performanceHat} and specialized pipelines. In the
domain of high-performance image processing,
Roly-Poly~\cite{ikarashi2021guidedOptimization} is a system built on top of the
Halide compiler~\cite{ragan2013halide} for expert engineers to explore
trade-offs and decide among possible optimizations. Odyssey is similar to
Roly-Poly in that it supports interactive workflows with an automated tool to
support expert users. In the statistical analysis domain, multiverse analysis
tools such as Boba~\cite{liu2020boba} and Multiverse
Debugger~\cite{gu2023understanding} enable expert statistical analysts to assess
the robustness and sensitivity of analysis results. The intended users are
experts in statistical analysis but not necessarily in multiverse authoring.
Similarly, Tsugite helps expert fabrication users create new wood
joints~\cite{larsson2020tsugite}. Odyssey adds to this growing body of research
on expert tools for a new domain, and we discuss key insights that 
could serve as design principles generalizable across domains (\autoref{sec:discussion}).

% on which there has been a big emphasis in the HCI literature, 