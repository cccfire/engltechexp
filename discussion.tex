\section{Discussion} \label{sec:discussion}
As the first expression rewriting workbench for the numerics community,
  % Odyssey shows that it is relatively easy to build useful expert tools
  Odyssey demonstrates how to build useful expert tools
  % that provide this user experience from the many existing tools that try to automate design search,
  % and here we offer general pointers from our experience.
  that enable users to more effectively search a design space. 
  Below, we discuss three insights that were key to Odyssey's design. 
  These insights serve as design principles that generalize to expert tools in other domains where users want to navigate a design space.

\paragraph{Expose heuristics, not states.}
  First, we found that exposing the internal exploration-focusing heuristics of the tool,
  rather than just the search states---%
  for Herbie, mainly the local error---helped users significantly,
  beyond its use in Herbie alone. %the original system.
By connecting this heuristic to other simpler metrics (like the input error plot), 
  users developed \textit{explanations} of the heuristic's value
  that helped them understand
  what was relevant about the search state---%
  for expression search,
  what subexpression was probably causing the error. 
By comparing the heuristic and their explanation across expressions, 
  users could check if the issue was solved, 
  even if the expression shape was too complicated for an automated tool to recognize.

  \paragraph{Give access to intermediate representations.}
  Second, we found that giving the user ownership over intermediate parts of the search
  made the tool much more useful. Doing so even allowed us to catch a bug in the underlying tool.
  A widely held belief among the HCI community is that higher levels of abstraction are more desirable for end-users. 
  Therefore, in an automated expert tool, it can seem natural to hide the middle of a search from the user to keep them working at a high level.
However, in our study, we found that users wanted to be able to see and control the search process. 
  Experts were particularly eager to introduce their own ideas and test assumptions.
In Odyssey, without building any separate tooling 
  except for a table that tracks candidates
  and synchronizes with visualizations of existing automated analyses, 
  users are able to explore a much broader space of possible solutions 
  in a way that was not possible with the original tool, 
  simply by letting a human manage search candidates.

\paragraph{Test expert workflows with relative novices}
% Finally, rather than preventing the user from thinking about the search, 
Finally, we were able to identify the appropriate level of abstraction in Odyssey because of our own iterative design process that involved novices and experts. 
  % the key to keeping our expert tool at the appropriate level of abstraction was establishing a workflow 
  % in a user study that included relative novices
  % so we understood the basic needs of and cognitive burden on experts.
  Involving novices sensitized us to the foundational cognitive burdens experts had developed workarounds for. 
% This ties back to the first point about showing a heuristic---%
%   even experts spent a lot of time 
%   and sometimes made mistakes predicting the heuristic (local error).
% However, we only identified this as a blocking issue 
%   after including novices in our initial design study.
We realized that if our tool could not help a novice at least understand basic issues,
  it was likely too opaque for experts to use productively.
% This change ended up being the most praised feature by experts.
The local error plot, a key feature we would not have included without involving novices, ended up being the most praised feature by experts.

% For Odyssey, the needs we found were classic debugging principles: 
%   the user needs to understand \textit{why} an expression has error (not just \textit{that} it has error)
%   and they ideally should see options for fixing the error.
% Surprisingly, the local error heatmap 
%   was originally only added to help out novices,
%   who had no hope of rewriting productively without it,
%   but it ended up being the most praised feature by experts.
% Without our initial user study involving novices, 
%   we doubt this feature would have been added.

% Within the particular domain of floating-point analysis, 
%   we see our results as very promising, especially for early work on supporting a
%   unified workflow.
% While we only included one source of automatic rewritings and one source of
%   analyses in Odyssey (both provided by our Herbie API), even this was enough for
%   users to understand error and solve real-world problems like the Rust
%   \texttt{asinh} bug we’ve discussed.
% Furthermore, the tool-independence of Odyssey’s database of expressions and
%   analyses means more tools can be easily added in the future.
\paragraph{Applications to other domains with user-driven design space search}
While this paper focuses on floating-point analysis, 
the above key insights and findings suggest generalizable principles for user-driven design space search. 
%  We see this work as having significance
%   for human-in-the-loop design space exploration
%   both within the domain of floating-point error correction 
%   and beyond.
The tool wrapped by Odyssey, Herbie, works in a way that should be 
familiar to anyone who has worked with a design space exploration tool
or classical AI search:
% We have said that the tool wrapped by Odyssey, Herbie, rewrites expressions.
% Under the hood, Herbie actually follows a familiar process:
  it identifies a troublesome part of an expression, 
  applies algebraic rewrites or approximations to that part of the expression to obtain new expressions, 
  tests those expressions to see if they are worth exploring, 
  and finally merges the best options.

The shape of this process 
  matches the workflow we describe for an analyst 
  identifying and solving problems with an expression step-by-step 
  while tracking possible rewriting directions.
% Broadly, this looks like a classical AI search, 
%   with a search frontier and one or more heuristic functions.
This search shape is used in tools across many domains,
  including in automated theorem provers, carpentry compilers,
  machining systems, and ASIC design space exploration tools. 
  % but the workflow we have described is often not the shape of the user experience of those tools. 
  % Yet, a wide range of expert tools do not apply the above three principles. 
  % As a result, the tools remain difficult to use and error-prone for even highly skilled users. 
  Yet, expert tools in these domains do not apply the above three principles. 
  As a result, the tools remain difficult to use and error-prone. 
  We hypothesize that applying the principles will improve expert tools in other domains where users search a design space. 


\subsection{Limitations and Future Work} 

A major limitation of our design process was the tight design loop 
  we had to maintain during development. 
While this was necessary to ensure
  we were building a system that would be useful to users, this meant
  we had to compromise on the polish of some features and 
  altogether avoid others which would take too long to implement or require
  disturbing many parts of the interface.
With more time, we plan to further improve the interface's layout and
  provide more structured expression editing support.

% \todo{Add other limitations, maybe
% tight design loop} For example, we were a little surprised to see users using the zoom function so
% frequently, despite it involving a relatively slow resampling operation that reanalyzes
% all of the expressions in the table. Based on this, we want to add support for using
% brushing on the graph to select and temporarily zoom on a subrange without
% resampling.

% The signs we saw and feedback we received pointed to this being an area that
%   can use much more interface-level support, especially from systems that
%   increase the interoperability and user-friendliness of analysis tools.
% Our experience was that even members of our team who originally had little
%   floating-point background were able to develop features users in our studies
%   ultimately found very helpful, simply by pushing for increased user access to
%   tool-internal data.

Of course, the main future work we have planned is to extend Odyssey to
  incorporate more analyses and sources of rewritings, including ideas like
  operation cost analyses and hardware-specific rewrites that were mentioned by
  the experts in our study.
Tools like PRECiSA~\cite{precisa} that already have an HTML-based analysis interface may be a good starting point 
  for testing these integrations. 

Floating-point experts were very appreciative of our work, and saw a variety of
  ways it could be extended to further support their particular areas of
  expertise.
These included ideas like adding support for multi-precision rewritings,
  incorporating operation cost analyses from Herbie and other tools, adding ways
  of helping human users simultaneously optimize at least 3 variables, and
  increasing support for splitting expressions into subregions and subexpressions
  based on domain-specific heuristics.

% There are many precedents for applying human-in-the-loop methods to the
%   solution of formal problems, as P3’s striking analogy between Odyssey and proof
%   assistants like Coq pointed out.
% Proof assistants are a well-explored domain with a clear interactive workflow,
%   and we think considering this analogy further could be also fruitful.

Odyssey also has clear potential application in floating-point education.
Several of our tasks asked users to explain to the interviewer potential
  problems with an expression using the interface, and both the experts and the
  novices in our formative study were able to point out areas of high error,
  select points, and zoom in to get a better look at problem regions to support
  diagnostic claims.
Odyssey has the potential to thrive in a classroom setting; it could be used by
  an instructor to show off how expression rewriting makes expressions more
  accurate or by students to explore and diagnose error sources an expression and
  try fixing them.
We plan to try applying Odyssey in an undergraduate class covering
  floating-point representations soon.

We are also excited by the explanatory potential offered by the incorporation
  of large language models (LLMs) like GPT.
We have found that available language models can, in fact, offer rewritings and
  generate plausible explanations for users, but they are prone to
  “hallucinating” and incorporating nonsensical logic, so their output must be
  validated before it is used.
With access to Odyssey's calculation and validation tools, 
  an LLM might be able to avoid these issues.
% Odyssey is the perfect platform for investigating the rewriting attempts of
% LLMs, and may be able to incorporate them in other ways, including by sending a
%   record of the user’s work on the analysis to the language model as context for
%   further generation.

Finally, a major possible extension was brought up independently by two
  different participants, who commented that they would be very interested in
  plugging in additional visualizations showing actual output effects of errors
  for each expression.
For example, one participant has worked with expressions representing ellipses,
  and wanted to see how different kinds of error could lead to distortion of the
  ellipses.
Allowing for additional visualizations would be a major possible improvement,
  since it will help users understand whether the error they see on the error
  plot matters when code is compiled and run in practice.
If (as with ellipses) the output space can be mapped back to specific input
  values, combining output visualization with the error graph heatmap will let
  experts relate points with noticeable error in the actual output to the
  particular mathematical operation causing that error.
% Thanks to the reactive database model of Odyssey’s codebase, it will be easy to
%   plug in this kind of output visualization and have it communicate with the rest
%   of the interface about selected points.

Overall, we are excited to see what floating-point experts and novices
  end up doing with Odyssey 
  and look forward to improving our support 
  for their work in the future.
