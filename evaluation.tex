\begin{table}
  \centering
  \begin{tabular}{|c|p{6cm}|}
\hline
  Expert \# &  Background: \\
\hline
    1 &  Industry, FP hardware + supercomputing (number systems for minimization problems), 45+ years. \\
\hline
    2 &  Professor, FP tools (mixed-precision conversions and program analysis), 10+ years. \\
\hline
    3 &  Grad student, FP hardware (datapath optimization), 5 years. 
    \\
\hline
    4 &  Professor, verification (correctness analysis), 9 years. \\
\hline
    5 &  Industry, FP hardware (interval analysis, transcendental functions), 3 years. 
    \\
\hline
\end{tabular}

  \caption {Five experts from the floating-point community evaluated and suggested future directions for our work.}
  \label{fig:experts}
\end{table}

\section{Expert Evaluation}
The goal of the expert evaluation
  was to assess the effectiveness of Odyssey
  in supporting the three-stage workflow we identified:
  diagnosing problems,
  generating solutions,
  and tuning expressions.
% (\autoref{sec:Workflow}). 

\subsection{Protocol}
We conducted an interview study
  with five experts from the floating-point community
  (see \autoref{fig:experts})
  to evaluate the effectiveness of Odyssey
  in supporting the three-stage workflow.
We recruited the experts via email
  through professional networks and communities (e.g., FPBench).
Each expert had different levels of experience
  in academia and industry,
  ranging from 3 years to over 45 years,
  and their backgrounds covered various aspects of floating-point systems,
  including hardware design, verification, and optimization.

Each interview session was conducted over Zoom, 
  with experts operating the tool via remote control 
  to avoid early issues we experienced with participants on networks
  with special configurations.
Interviews lasted between 60 and 90 minutes
  and consisted of three parts:
\begin{itemize}
  \item \textit{Introduction and tutorial.} 
The first author briefly introduced Odyssey 
  and the problems it is designed to address. 
Then, each expert followed a hands-on tutorial 
  demonstrating the usage of Odyssey on a simple example.
\item \textit{Seven tasks.} Each expert completed seven tasks, each designed
for one of the three workflow stages and aimed at eliciting the experts'
reactions to different parts of Odyssey's interface (\autoref{fig:tasks}). If the experts
encountered difficulties, the first author provided guidance or reminded
them of relevant interface features from the tutorial.
\item \textit{Exit survey and discussion.} To conclude, each expert
completed a survey (\autoref{fig:survey-results}) 
and participated in a semi-structured interview with the
first author, where the experts reflected on their experience with Odyssey
and provided feedback on potential improvements and extensions. The first
author specifically asked for experts' opinions on the legitimacy of the
workflow we aim to support, its relevance to their work, and the extent to
which they felt Odyssey supported each part of the workflow.
\end{itemize}
Throughout all three parts, the experts' screens and audio were recorded. The
first author also took note of the experts' comments, insights, and responses to
the tasks and survey questions. All study materials are provided as supplemental material.

\subsection{Analysis and Results}
We conducted an iterative, thematic analysis of expert solutions and the first
author's notes for each stage of the workflow. Below, we discuss the experts'
responses to the relevant tasks and survey items for each stage. Through this
analysis, we aim to provide a qualitative evaluation of Odyssey's effectiveness
in supporting each part of the workflow.

\begin{table*}
  \centering
  \begin{tabular}{|c|p{5cm}|p{6cm}|c|}
  \hline
      Task & Description & Targeted part of workflow & Success rate \\
  % \hline
  % A & Given floating-point implementation (I), $log(x + \sqrt{x \cdot x + 1})$, for the inverse hyperbolic 
  % sine, identify the parts of the expression causing errors. & Identify troublesome 
  % subexpressions and problematic ranges. & 4/5 \\
  % \hline
  % B & Use Odyssey to analyze the troublesome square root subexpression from I. 
  % & Gather solutions for subexpression and use these to optimize original expression.
  % & 4/4 \\
  % \hline
  % C & Identify problems with branch expressions in automated solutions for I.
  % & Judge tool-generated expressions using personal intuition. & 3/4\\
  % \hline
  % D & Use Odyssey to find and recommend \textit{log1p}.
  % & Incorporate accumulated insights to nudge Herbie past roadblocks.
  % & 2/3 \\
  % \hline
  % E & Evaluate whether the solution for I after Tasks A-D are trustworthy.
  % & Use Odyssey's feedback on expressions and information about expression soundness to 
  % evaluate expressions' trustworthiness and fitness based on personal standards.
  % & 2/2 \\
  % \hline
  % F & Use Odyssey to optimize the floating-point implementation
  % $(exp(x)-2) + exp(-x)$ using branch expressions. & Mix solutions from different sources
  % to create strong solutions. 
  % & 3/4 \\
  % \hline
  \hline
  1 & $log(x + \sqrt{x \cdot x + 1})$ is an expression for the inverse hyperbolic 
  sine. Identify the parts of the expression causing errors for large/small $x$. & Diagnose troublesome 
  subexpressions and problematic ranges. & 4/5 \\
  \hline
  2 & Use Odyssey to find a solution for the troublesome square root subexpression from task 1. 
  & Generate solutions for the subexpression and use these to optimize original expression.
  & 4/4 \\
  \hline
  3 & Is your solution to task 2 good enough? 
  & Use visualizations to form evaluation criteria for ending analysis.
  & 4/4 \\
  \hline
  4 & Identify problems with branch expressions in fully automated solutions for task 1.
  & Explain important features of expressions and diagnose issues. & 3/4\\
  \hline
  5 & Use Odyssey to find and recommend \textit{log1p} to solve small $x$.
  & Nudge an automated tool past roadblocks to generate better solutions.
  & 2/3 \\
  \hline
  6 & Evaluate whether the full solution for the expression after tasks 1-5 is trustworthy.
  & Use Odyssey's feedback on expressions and information about expression soundness to 
  evaluate expressions' trustworthiness and fitness based on personal standards.
  & 2/2 \\
  \hline
  7 & Use branch conditions to outperform a fully automated rewriting for the expression
  $(exp(x)-2) + exp(-x)$. & Mix solutions from different sources and tune branch conditions
  to create stronger solutions. 
  & 3/4 \\
  \hline
\end{tabular}

  \caption {Experts worked through up to seven tasks to exercise the features of Odyssey before a survey-based discussion. Due to time constraints, 
  not all experts completed all tasks.}
  \label{fig:tasks}
\end{table*}

\subsubsection*{First Stage: Diagnosing Problems}

Task 1 required experts to analyze the error in an inverse hyperbolic sine
implementation and identify the parts of the expression causing errors, then
decide which operation or operations needed to be rewritten in order for the
rewriting to correctly handle large inputs. Among the five experts, four
successfully completed this task, relying on Odyssey's error visualizations
(see~\autoref{fig:diagnosis}).

P2 explored multiple input ranges in order to identify 
  the two problematic operations: 
  \longquote{This is across the
entire sample... so I wonder if it's doing something different on this side
[clicking a point with a small $x$ value and looking at the local error graph] So
there it's all the log, and over there... [clicks a large $x$ value] it's all the
square root. So that's interesting, it's actually coming from different
operations.} Here, the error plot effectively surfaced the two areas of high
error (small and large $x$ values), giving the expert clear places to look for
troublesome operations. Then, by switching between inputs in different regions,
the expert was able to see that the problematic operation was different
between these regions.

In the survey (see item 3 in~\autoref{fig:survey-results}),
  all five experts rated the interface's ability
  to help identify or confirm specific problems with expressions 
  at a 7 out of 7.
We attribute this success mainly to the error plot and local error heatmap,
  which implemented the second principle we identified for a good diagnosis tool.
  They supported the user in assigning responsibility for error
  without relying on expertise or resorting to trial and error.
% As P3 explained, in the real-world examples we worked on, 
%   \longquote{[the error characteristics] are varying wildly across the domain 
%   so that you don't get that much from an average, 
%   so the ability to click on points 
%   and identify the error for a subdomain is super useful.}
As P2 concluded, \longquote{%
  Having the graph and being able to click 
  on the different places where error is high 
  is definitely nicer than just looking at output 
  in a text file.}

\subsubsection*{Second Stage: Generating Solutions}

We designed several tasks to evaluate Odyssey's support for collecting and
evaluating new expressions that address the identified problems in floating-point
expressions. Close to all experts who attempted each
task succeeded (see Tasks 2 and 5 in~\autoref{fig:tasks}).

Task 2 required experts to analyze a troublesome
subexpression from Task 1 and find a better rewriting for it. Then, experts needed
to bring the solution back to the original analysis and decide if they were
happy with it. Four of the five experts who attempted Task 2 successfully
completed it, showing that the interface facilitated the collection of solutions
and their integration into existing expressions. Of those four, two experts found their own unique approaches to solving the problem identified in Task 1 rather than relying on an automated solution. One expert pulled a factor of $x$ out of the square root, and another expert created a branch that switched to an approximation for large values of $x$. Both of these approaches showed low error on the error plot, though the experts noted there could be issues with these choices (for example, branching impacts performance, and dividing by $x$ is risky when $x$ could be 0). This showcases the flexibility of
Odyssey in allowing users to explore alternative solutions and evaluate their
impact on the error plot. 
(The expert who did not complete Task 2 was our first participant,
with whom we lost much of the interview time due to the networking issues mentioned earlier.)

Similarly, Task 5 asked experts to find a more accurate rewriting for a
subexpression applicable to small values of $x$. Three out of the four
experts successfully completed this task, further supporting the
effectiveness of Odyssey in assisting experts in gathering and evaluating
potential solutions.

P3 had the following to say about working through the process up to Task 5:
\longquote{%
It feels like quite a natural way you might approach this problem as a human. You're burrowing down into it more precisely and pushing your error around a little bit. I thought the transition of `we've moved the error from the log into the subtract [using log1p], now I know
how to deal with the error in a subtract as well' felt natural, ... since ... once we figure out it was going to be the subtract that was giving us trouble, then 
[we can use Herbie to rewrite successfully]. 
It gets there much faster, 
but it's cool that I also feel that 
I would have thought about going in a similar direction.}

% Task 4 asked experts to examine the best solutions recommended by Herbie,
% which all contained conditional branches, and identify any issues despite the
% relatively good average error. experts needed to refer to the expression
% details area and examine the branch point regions on the graph to pinpoint areas
% with high remaining error and explain that the branch points were still risk
% spots for error. Out of the four experts who attempted Task 4, three
% successfully completed it, indicating that the interface mostly enabled users to
% effectively evaluate and analyze expression features. This result demonstrates
% that Odyssey proficiently supports the second step of our workflow, although
% there is still room for improvement.

In the survey, experts rated the interface's ability to generate ideas for
solving specific problems (item 4) with scores ranging from 5 to 7, with an
average of 5.8. The interface's effectiveness in evaluating the quality of ideas
quickly (item 5) was rated between 5 and 7, with an average of 6.4. These
relatively high ratings indicate that the experts found Odyssey helpful in
generating and evaluating ideas for improving floating-point expressions.

Users were able to use Odyssey to successfully generate 
  a variety of valid nontrivial new expressions for analysis, 
  both using an automated tool 
  (e.g. the way we expected users to solve Task 2) 
  and by themselves (P5 and P4). This was significantly different from our experience 
  in the earliest parts of the design process.
  The ability to send rewrites back to Herbie was a vital
  part of the solution generation process for the three experts who were able to complete Task 5.

% Our results for this stage 
%   follow from the three design principles 
%   for solution generation support 
%   that we identified through our design process.
% The main result 
%   was that users were able to use Odyssey to successfully generate 
%   a variety of valid nontrivial new expressions for analysis, 
%   both using an automated tool 
%   (e.g. the way we expected users to solve Task 2) 
%   and by themselves (P5 and P4).
% This was significantly different from our experience 
%   in the earliest parts of \phaseFour~of the design process,
%   which we attribute to improved diagnosis tools as well as
%   providing immediate feedback on syntax errors 
%   and rendering a LaTeX version of the user's in-progress expression.
% Finally, the ability to send rewrites back to Herbie was a vital
%   part of the solution generation process for the three experts who were able to complete task 5.


\begin{table*}
  \centering
  {\renewcommand{\arraystretch}{1.25}
\begin{tabular}{||c|p{12cm}|c|c||}
    \hline
    \# & Survey Questions: & Results: & Average:\\
    \hline
    1 & "The workflow made sense to me and I was able to follow it." 
    & 5,  5,  6,  7,  7 & 6/7\\
    \hline
    2 & "This workflow matches my experience approaching real numerical analysis
    problems." & 4,  6,  6,  6,  6 & 5.6/7\\
    \hline
    3 & "The interface helped me identify or confirm specific problems with
    expressions." & 7,  7,  7,  7,  7 & 7/7\\
    \hline
    4 & "The interface allowed me to generate ideas for solving a specific problem." & 5,  5,  6,  6,  7 & 5.8/7\\
    \hline
    5 & "The interface let me evaluate the quality of ideas for rewritings quickly."
    & 5,  6,  7,  7,  7 & 6.4/7\\
    \hline
    6 & "It was easy to compare expressions in the interface." & 6,  6,  6,  7,  7 & 6.4/7\\
    \hline
    7 & "It was easy to mix together expressions from different sources in the interface." & 4,  5,  5,  6,  7 & 5.4/7\\
    \hline
    8 & "The interface let me focus on thinking about the problem at a high level." & 5,  6,  7,  7,  7 & 6.4/7 \\
    \hline
    9 & "I can think of ways to extend this workflow + interface to address numerical analysis problems that I have worked on." & 5,  6,  7,  7,  7 & 6.4/7 \\
    \hline
\end{tabular}}

  \caption {After completing the seven tasks, experts were asked to evaluate
 different aspects of the tool on a scale of 1 to 7. }
  \label{fig:survey-results}
\end{table*}

\subsubsection*{Third Stage: Tuning}

The third stage of our proposed workflow involves tuning expressions to further
optimize their accuracy and performance. To assess Odyssey's support for this
stage, we evaluated Task 7, as well as survey items 6 and 7, which inquired about
the interface's support for comparing and mixing different expressions.

Task 7 challenged experts to create a more accurate expression than
Herbie's best alternative for a given expression by combining different
solutions and fine-tuning the branch point. The task demonstrated that a human
can use Odyssey to outperform Herbie's internal heuristics when unique
requirements call for a tailored approach. 
After using the range zoom feature and noticing 
  Herbie's solution was still outperforming their solution 
  on a small region,
  P2 remarked, 
  \shortquote{So in this view, we can see that 
  we don't have quite the right number [for the branch point].} 
The expert then adjusted the branch point 
  based on the visual feedback.

In the survey, experts rated the interface's capacity to help them mix expressions from different sources (item 7) with
scores ranging from 4 to 7, with an average of 5.4. The interface's support for
comparing different expressions (item 6) was rated even more highly, at an
average of 6.4 (range from 6 to 7).

As we can see in the example above, 
  the especially high rating for comparison
  was likely a result of combining the ability 
  to plot the error for different expressions
  together with zooming to focus on getting feedback on specific regions.
A couple experts (P4, P5)
  mentioned wanting more support for combining expressions, 
  especially around conditional branches.
P4 explained that 
  an automated tool might be able to add guard conditions 
  where appropriate.

Finally, the experts appreciated the potential power 
  of mixing human and automated solutions, 
  with P3 commenting that 
  suggesting \texttt{log1p} and \texttt{hypot} to Herbie 
  felt similar to proof assistant tools where \shortquote{if you just add in an additional step on the way or an additional lemma... then it can actually nudge it over that threshold.}

In summary, the results from Task 7, along with the survey responses for items 6
and 7, provide evidence that Odyssey effectively supports tuning expressions for optimal accuracy and
performance. The interface enables users to mix expressions
and adjust coefficients while offering real-time feedback, streamlining the
tuning process and enhancing the overall quality of floating-point expressions.
