In recent years, researchers have proposed a number of automated tools to
  identify and improve floating-point rounding error in mathematical expressions.
However, users struggle to effectively apply these tools.
In this paper, we work with 
  novices, experts, and tool developers to 
  investigate user needs during the expression rewriting process.
We find that users follow an iterative design process.
They want to compare expressions on multiple input ranges, 
  integrate and guide various rewriting tools,
  and understand where errors come from.
We organize this investigation's results into a three-stage workflow
  and implement that workflow
  in a new, extensible workbench dubbed Odyssey.
Odyssey enables users to:
  (1) \textit{diagnose} problems in an expression, 
  (2) \textit{generate solutions} automatically or by hand, and 
  (3) \textit{tune} their results. 
Odyssey tracks a working set of expressions
  and turns a state-of-the-art automated tool ``inside out,''
  giving the user access to internal heuristics, algorithms, 
  and functionality.
In a user study, Odyssey enabled five expert numerical analysts
   to solve challenging rewriting problems
  where state-of-the-art automated tools fail.
In particular, the experts unanimously praised Odyssey’s novel support for
  interactive range modification and local error visualization.